\section{Conclusion}\label{conclusion}
Finally, it can be said that the approach presented in this study can be considered as a basis for future research to further investigate the relevant cognitive processes involved in collaborative spatial problem solving. This study was intentionally designed to be very controlled and therefore does not compare to any application of real life problem solving. The goal was to design a task that allows to quantify performance in spacial problem solving tasks. Also the study proved virtual reality to be a well suited method of research for the field of collaborative spatial problem solving. The main finding was that groups perform significantly better than individuals in the designed spatial problem solving task. Further research will have to show if the same applies for more complex variations of the task and how important factors like perspective and communication are.