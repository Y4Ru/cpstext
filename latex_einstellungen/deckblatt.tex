% das Papierformat zuerst
%\documentclass[a4paper, 11pt]{article}

% deutsche Silbentrennung
%\usepackage[ngerman]{babel}

% wegen deutschen Umlauten
%\usepackage[ansinew]{inputenc}

% hier beginnt das Dokument
%\begin{document}


\thispagestyle{empty}

%\begin{figure}[t]
% \includegraphics[width=0.6\textwidth]{abb/fh_koeln_logo}
%\end{figure}


\begin{verbatim}


\end{verbatim}

\begin{center}
\Large{Universität Tübingen}\\
\Large{- Mathematisch-Naturwissenschaftliche Fakultät -}\\
\end{center}


\begin{center}
\Large{Masterarbeit}
\end{center}
\begin{verbatim}




\end{verbatim}
\begin{center}
\doublespacing
\textbf{\LARGE{Spatial problem solving and collaboration in virtual reality}}\\
\singlespacing
\begin{verbatim}

\end{verbatim}
\textbf{im Studiengang Kognitionswissenschaft}
\end{center}
\begin{verbatim}

\end{verbatim}
\begin{center}

\end{center}
\begin{verbatim}

\end{verbatim}
\begin{center}
\textbf{zur Erlangung des akademischen Grades Master of Science}
\end{center}
\begin{verbatim}






\end{verbatim}
\begin{flushleft}
\begin{tabular}{llll}
\textbf{Autor:} & & Marcel Bechtold& \\
& & MatNr. 3949100 & \\
& & \\
\textbf{Version vom:} & & \today &\\
& & \\
\textbf{1. Betreuerin:} & & Prof. Dr. Enkelejda Kasneci &\\
\textbf{2. Betreuer:} & & Dr. Tobias Meillinger &\\
\end{tabular}
\end{flushleft}