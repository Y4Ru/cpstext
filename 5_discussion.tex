\section{Discussion}\label{discussion}

Designing a task for spatial problem solving is quite complex and many factors have to be taken into consideration. In order to achieve meaningful results from experiments that are based on such a task, it is important that the task is very controlled. When it comes to comparing individuals and groups in problem solving it must be made sure that the task can be executed in a similar manner both in the individual and group condition. We decided to position participants in the individual condition in front of the task having the starting cube on their side of the solution space. In the group condition we positioned the two participants on two sides of the problem space facing each other. Therefore only one participant in the group condition can have the same position as in the individual condition. The other participant stands on the other side of the problem space having a completely different perspective of the problem. It is intended that participants have different perspectives of the problem in the group condition, because perspective is a relevant factor in spatial problem solving - especially when it comes to collaboration. \cite{Amorim2003} In natural settings two or more people working on the same problem usually do not have the same perspective and therefore have a cost in communicating the differences in perspective and must to try create a shared representation (\cite{Roschelle1995} and \cite{Frankenstein2012}) of the current state of the problem on which they are working together. In the collaborative condition participants have the same costs of communicating the differences in perspective. Nevertheless, it could be found that the beneficial aspects of having two different perspectives and being able to work together proved to prevail the costs of communication and shared representation. This is assumed to be the main reason why groups performed much better than individuals.

Another benefit of working together on the task, was that participants were able to simultaneously work on different sub processes of the problem. While one participant identifies the required colors for the next slot, the other participant can already start searching in the remaining cubes and maybe sort them into relevant or irrelevant cubes based on the current colors that are already part of the solution space.

Since the task was very controlled and did not leave too many options of how to transition from the initial state to the goal state, the question arises if the group condition would have also performed this much better if there would have been no information about the sequence of putting the cubes into the solution space. This could be easily tested in a follow up study, in which the information about the sequence is omitted. All other aspects of the experiment could remain the same.

In another follow up study the advantage of virtual reality to put both participants in the same position could be used in order to find out how much of an effect a shared perspective has on the performance. It is difficult thought to have two people interact from the exact same virtual location with a virtual environment since collisions and conflicts in actions are more likely to occur. This variation also obviously can not occur in a real live scenario at all, but comparing that variation to the results presented in the current study may give interesting insights on how relevant perspective is for collaborative spacial problem solving.

Furthermore the combination of virtual reality with eye tracking could help to better understand the cognitive processes involved in collaborative spatial problem solving when it comes to action planning and execution. 